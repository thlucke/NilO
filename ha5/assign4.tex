\assignment{5.4}

Betrachte das Optimierungsproblem $\begin{cases}\min\limits_{x\in\mathbb R^2} f(x)=(x_1-1)(x_2-1)=x_1x_2-x_1-x_2+1\\x_1+x_2\leq 2\\x_1\geq 0\\x_2\geq 0\end{cases}$\newline\newline
\begin{compactenum}[(i)]
\item Der Gradient von $f$ lautet $$\nabla f(x)=\begin{bmatrix}x_2-1\\x_1-1\end{bmatrix}.$$
Die Nebenbedingungen lassen sich umschreiben zu
\begin{align*}
\begin{bmatrix}1&1\\-1&0\\0&-1\end{bmatrix}\begin{bmatrix}x_1\\x_2\end{bmatrix}&\leq \begin{bmatrix}2\\0\\0\end{bmatrix}
\end{align*}
Die Karush-Kuhn-Tucker-Bedingungen für das Optimierungsproblem sind also
\begin{align*}\begin{bmatrix}x_2+\mu_a-\mu_b-1\\x_1+\mu_a-\mu_c-1\end{bmatrix}=\begin{bmatrix}x_2-1\\x_1-1\end{bmatrix}+\begin{bmatrix}1&1\\-1&0\\0&-1\end{bmatrix}^T\begin{bmatrix}\mu_a\\\mu_b\\\mu_c\end{bmatrix}&=0\\
(x_1+x_2-2)\mu_a-x_1\mu_b-x_2\mu_c=
\left\langle\begin{bmatrix}1&1\\-1&0\\0&-1\end{bmatrix}\begin{bmatrix}x_1\\x_2\end{bmatrix}- \begin{bmatrix}-2\\0\\0\end{bmatrix} ,\begin{bmatrix}\mu_a\\\mu_b\\\mu_c\end{bmatrix} \right\rangle &=0\\\mu_a,\mu_b,\mu_c&\geq 0\end{align*}
Einsetzen des zulässigen Punktes $\begin{bmatrix}x_1\\x_2\end{bmatrix}=\begin{bmatrix}0\\2\end{bmatrix}$ liefert das lineare Gleichungssystem 
\begin{align*}\begin{bmatrix}1&-1&0\\1&0&-1\\0&0&-2\end{bmatrix}\begin{bmatrix}\mu_a\\\mu_b\\\mu_c\end{bmatrix}&=\begin{bmatrix}-1\\1\\0\end{bmatrix}\\\mu_a,\mu_b,\mu_c&\geq 0\end{align*}
mit der eindeutigen Lösung $\begin{bmatrix}\mu_a\\\mu_b\\\mu_c\end{bmatrix}=\begin{bmatrix}1\\2\\0\end{bmatrix}$.\\\\
Also erfüllt $\begin{bmatrix}x_1\\x_2\end{bmatrix}=\begin{bmatrix}0\\2\end{bmatrix}$ die Karush-Kuhn-Tucker-Bedingungen mit dem eindeutigen Lagrange-Multiplikator $\begin{bmatrix}\mu_a\\\mu_b\\\mu_c\end{bmatrix}=\begin{bmatrix}1\\2\\0\end{bmatrix}$. Analog erfüllt $\begin{bmatrix}x_1\\x_2\end{bmatrix}=\begin{bmatrix}2\\0\end{bmatrix}$ die Karush-Kuhn-Tucker-Bedingungen mit dem eindeutigen Lagrange-Multiplikator $\begin{bmatrix}\mu_a\\\mu_b\\\mu_c\end{bmatrix}=\begin{bmatrix}1\\0\\2\end{bmatrix}$.\newpage
\item Die Hesse-Matrix von $f$ lautet unabhängig von $x$
$$f''(x)=\begin{bmatrix}0&1\\1&0\end{bmatrix}$$
Die hinreichende Bedingung zweiter Ordnung für das Optimierungsproblem ist also
\begin{align*}&\exists \alpha\geq 0: \forall d=\begin{bmatrix}d_1\\d_2\end{bmatrix}\in \{d\in K(\mathcal F, x)\mid\langle \nabla f(x),d\rangle=0\}:\\ &2d_1d_2=d^T\begin{bmatrix}0&1\\1&0\end{bmatrix}d=d^Tf''(x)d\geq\alpha\|d\|^2=d_1^2+d_2^2\end{align*}
Die letzte Ungleichung ist äquivalent zu
$$(d_1-d_2)^2=d_1^2+d_2^2-2d_1d_2\leq 0$$
Für $x=\begin{bmatrix}0\\2\end{bmatrix}$ ist $\nabla f(x)=\begin{bmatrix}1\\-1\end{bmatrix}$. Für $d_1-d_2=\langle \nabla f(x),d\rangle=0$ ist also $(d_1-d_2)^2\leq 0$ und damit die hinreichende Bedingung zweiter Ordnung erfüllt.\\\\
Für $x=\begin{bmatrix}2\\0\end{bmatrix}$ ist $\nabla f(x)=\begin{bmatrix}-1\\1\end{bmatrix}$. Für $d_2-d_1=\langle \nabla f(x),d\rangle=0$ ist also $(d_1-d_2)^2\leq 0$ und damit die hinreichende Bedingung zweiter Ordnung erfüllt.
\end{compactenum}