\assignment{2.2}

Gegeben sei $z\in\R^n$. Wir untersuchen zu $\lambda\geq 0$
\begin{displaymath}
 f_\lambda\colon \R^n\to\R,\quad f_\lambda(x)=\frac{1}{2}\Vert x-z\Vert_2^2+\lambda\Vert x\Vert_1=\frac{1}{2}\sum_{k=1}^n |x_k-z_k|^2+\lambda\sum_{k=1}^n|x_k|.
\end{displaymath}
\begin{compactenum}[(i)]
 \item Wir zeigen, dass für jedes $\lambda\geq 0$ ein eindeutiges $x_\lambda = \arg\min_{x\in\R^n}f_\lambda(x)$ gibt. Sei also $\lambda\geq 0$ beliebig.
 Wir zeigen, dass $f_\lambda$ koerziv und streng konvex ist. Mit
 \begin{displaymath}
  f_\lambda(x)=\frac{1}{2}\Vert x-z\Vert_2^2+\lambda\Vert x\Vert_1\geq \frac{1}{2}\Vert x-z\Vert_2^2\geq \Vert x\Vert_2^2-\Vert x\Vert_2\Vert z\Vert_2+\Vert z \Vert_2^2\geq \Vert x\Vert_2(\Vert x\Vert_2-\Vert z\Vert_2)
 \end{displaymath}
 folgt die Koerzitivität von $f_\lambda$.\\
 Für den Nachweis der Konvexität seien $x,y\in\R^n,\,x\neq y$ und $t\in(0,1)$. Mit der Parallelogrammgleichung gilt
 \begin{align*}
  &f_\lambda(tx+(1-t)y)&=&\frac{1}{2}\Vert tx+(1-t)y-z\Vert_2^2+\lambda\Vert tx+(1-t)y\Vert_1\\
  &&\leq& \frac{1}{2}\Vert tx+(1-t)y-tz-(1-t)z\Vert_2^2+\lambda(t\Vert x\Vert_1+(1-t)\Vert y\Vert_1)\\
  &&=&\frac{1}{2}\Vert t(x-z)+(1-t)(y-z)\Vert_2^2+\lambda(t\Vert x\Vert_1+(1-t)\Vert y\Vert_1)\\
  &&=&\Vert t(x-z)\Vert_2^2+\Vert(1-t)(y-z)\Vert_2^2\\&&&-\frac{1}{2}\Vert t(x-z)-(1-t)(y-z)\Vert_2^2 +\lambda(t\Vert x\Vert_1+(1-t)\Vert y\Vert_1)\\
  &&\leq&\Vert t(x-z)\Vert_2^2+\Vert(1-t)(y-z)\Vert_2^2 +\lambda(t\Vert x\Vert_1+(1-t)\Vert y\Vert_1)\\
  &&=&t^2\Vert x-z\Vert_2^2+\lambda t\Vert x\Vert_1+(1-t)^2\Vert(y-z)\Vert_2^2 +\lambda(1-t)\Vert y\Vert_1\\
  &&<&\frac{t}{2}\Vert x-z\Vert_2^2+\lambda t\Vert x\Vert_1+\frac{(1-t)}{2}\Vert(y-z)\Vert_2^2 +\lambda(1-t)\Vert y\Vert_1\\
  &&=&t f_\lambda(x)+(1-t) f_\lambda(y).
 \end{align*}
 Insgesamt ist $f_\lambda$ also streng konvex und somit insbesondere stetig sowie koerziv und besitzt damit einen eindeutigen Minimierer $x_\lambda$.
 \\
%  \item Sei $\lambda\geq 0$ beliebig. Dann sind die Richtungsableitungen an der Stelle $x$ in Koordinatenrichtung $e_k,\, k=1,\dots, n$ gegeben durch
%  \begin{displaymath}
%   f'(x,e_k)=(x_k-z_k)+\lambda \sgn(x_k).
%  \end{displaymath}
%  Für die Erfüllung der Variationsungleichung fordern wir weiter
%  \begin{displaymath}
%   0\leq   f'(x,e_k)=(x_k-z_k)+\lambda \sgn(x_k)
%  \end{displaymath}


\end{compactenum}

