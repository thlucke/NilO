\assignment{2.2}

Gegeben sei $z\in\R^n$. Wir untersuchen zu $\lambda\geq 0$
\begin{displaymath}
 f_\lambda\colon \R^n\to\R,\quad f_\lambda(x)=\frac{1}{2}\Vert x-z\Vert_2^2+\lambda\Vert x\Vert_1=\frac{1}{2}\sum_{k=1}^n |x_k-z_k|^2+\lambda\sum_{k=1}^n|x_k|.
\end{displaymath}
\begin{compactenum}[(i)]
 \item Wir zeigen, dass für jedes $\lambda\geq 0$ ein eindeutiges $x_\lambda = \arg\min_{x\in\R^n}f_\lambda(x)$ gibt. Sei also $\lambda\geq 0$ beliebig.
 Wir zeigen, dass $f_\lambda$ koerziv und streng konvex ist. Mit
 \begin{displaymath}
  f_\lambda(x)=\frac{1}{2}\Vert x-z\Vert_2^2+\lambda\Vert x\Vert_1\geq \frac{1}{2}\Vert x-z\Vert_2^2\geq \Vert x\Vert_2^2-\Vert x\Vert_2\Vert z\Vert_2+\Vert z \Vert_2^2\geq \Vert x\Vert_2(\Vert x\Vert_2-\Vert z\Vert_2)
 \end{displaymath}
 folgt die Koerzitivität von $f_\lambda$.\\
 Auf der anderen Seite ist $f_\lambda$ Summe einer streng konvexen Funktion $n: x\mapsto \Vert x-z\Vert_2^2$ und einer konvexen Funktion $x\mapsto \Vert x\Vert_1$ und
 somit selbst streng konvex. Bemerke dazu, dass $n''(x)=\id_{\R^n}$ positiv definit ist und daher $n$ streng konvex und dass für alle $x,y\in\R^n$ und $t\in [0,1]$
 mit der Dreiecksungleichung
 \begin{displaymath}
  \Vert tx+(1-t)y\Vert_1\leq t\Vert x\Vert_1+(1-t)\Vert y\Vert_1
 \end{displaymath}
 gilt.
 Insgesamt ist $f_\lambda$ also streng konvex und somit insbesondere stetig sowie koerziv und besitzt damit einen eindeutigen Minimierer $x_\lambda$.
 \\
 \item
%  \item Sei $\lambda\geq 0$ beliebig. Dann sind die Richtungsableitungen an der Stelle $x$ in Koordinatenrichtung $e_k,\, k=1,\dots, n$ gegeben durch
%  \begin{displaymath}
%   f'(x,e_k)=(x_k-z_k)+\lambda \sgn(x_k).
%  \end{displaymath}
%  Für die Erfüllung der Variationsungleichung fordern wir weiter
%  \begin{displaymath}
%   0\leq   f'(x,e_k)=(x_k-z_k)+\lambda \sgn(x_k)
%  \end{displaymath}


\end{compactenum}

