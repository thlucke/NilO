\assignment{2.2}

Gegeben sei $z\in\R^n$. Wir untersuchen zu $\lambda\geq 0$
\begin{displaymath}
 f_\lambda\colon \R^n\to\R,\quad f_\lambda(x)=\frac{1}{2}\Vert x-z\Vert_2^2+\lambda\Vert x\Vert_1=\frac{1}{2}\sum_{k=1}^n |x_k-z_k|^2+\lambda\sum_{k=1}^n|x_k|.
\end{displaymath}
\begin{compactenum}[(i)]
 \item Wir zeigen, dass es für jedes $\lambda\geq 0$ ein eindeutiges $x_\lambda = \arg\min_{x\in\R^n}f_\lambda(x)$ gibt.\\\\
Sei also $\lambda\geq 0$ beliebig.
 Wir zeigen, dass $f_\lambda$ koerzitiv und streng konvex ist. Mit
 \begin{displaymath}
  f_\lambda(x)=\frac{1}{2}\Vert x-z\Vert_2^2+\lambda\Vert x\Vert_1\geq \frac{1}{2}\Vert x-z\Vert_2^2\geq \Vert x\Vert_2^2-\Vert x\Vert_2\Vert z\Vert_2+\Vert z \Vert_2^2\geq \Vert x\Vert_2(\Vert x\Vert_2-\Vert z\Vert_2)
 \end{displaymath}
 folgt die Koerzitivität von $f_\lambda$.\\\\
Außerdem ist $f_\lambda$ Summe der streng konvexen Funktion $n: x\mapsto \Vert x-z\Vert_2^2$ und der konvexen Funktion $x\mapsto \Vert x\Vert_1$. Bemerke dazu, dass $n''(x)=\id_{\R^n}$ positiv definit und $n$ daher streng konvex ist und dass für alle $x,y\in\R^n$ und $t\in [0,1]$
wegen der Dreiecksungleichung
 \begin{displaymath}
  \Vert tx+(1-t)y\Vert_1\leq t\Vert x\Vert_1+(1-t)\Vert y\Vert_1
 \end{displaymath}
gilt. Also ist $f_\lambda$ selbst streng konvex und somit insbesondere stetig. Aus der Stetigkeit und Koerzitivität folgt, dass $f_\lambda$ einen eindeutigen Minimierer $x_\lambda$ besitzt.
\item Nach (i) ist der Minimierer $x_\lambda=\underset{y\in\R^n}{\text{argmin}} f_\lambda(x)$ eindeutig bestimmt.
Löst ein $x\in\mathbb R^n$ nun die Variationsungleichung, gilt also $f'(x,h)\geq 0$ für alle $h\in\R^n$, so gilt $x=x_\lambda$.\\\\
Dann gilt insbesondere $f'(x_\lambda,h)\geq 0$ für jeden k-ten Einheitsvektoren $e_k=(\delta_{ik})_{i=1,...,n}$ und ihre additiven Inversen $-e_k=(-\delta_{ik})_{i=1,...,n}$.\\\\
Mit $x_\lambda=(x_k)_{k=1,...,n}$ gilt für $k=1,...,n$ also:
$$f_\lambda'(x_\lambda,e_k)=\lim\limits_{t\searrow 0}\frac{f(x_\lambda+te_k)-f(x_\lambda)}{t}=(x_k-z_k)+\lambda\sgn(x_k)\geq 0$$
und
$$f_\lambda'(x_\lambda,-e_k)=\lim\limits_{t\searrow 0}\frac{f(x_\lambda-te_k)-f(x_\lambda)}{t}=-((x_k-z_k)+\lambda\sgn(x_k))\leq 0.$$
Aus beiden Ungleichungen ergibt sich die Gleichung
$$x_k-z_k+\lambda\sgn(x_k)=0$$
und durch Umstellen die implizite Identität
$$x_k=z_k-\lambda\sgn(x_k).$$
Ist $z_k>\lambda$, so folgt $\sgn(x_k)=1$, da für $x_k<0$ der Widerspruch $0>x_k=z_k+\lambda>0$ entsteht. Ist $z_k<\lambda$, so folgt analog $\sgn(x_k)=-1$. Für $z_k=\lambda$ gilt aus ähnlichen Gründen $\sgn(x_k)=0$, also $x_k=0$. Daher erhalten wir sogar die explizite Formel
$$x_k=z_k-\lambda\sgn(z_k-\lambda).$$\\\\
Bemerkung:\\\\
Sei nun $|z_k|\leq \lambda$, d.h. $z_k<\lambda$ und $-z_k<\lambda$. Aus $z_k<\lambda$ folgt wie oben $\sgn(x_k)=-1$ da sonst $0>x_k=z_k+\lambda>0$ zum Widerspruch führt.  Es folgt also $\sgn(x_k)=0$ und damit $x_k=0$. Also kommen für große $\lambda$ dünnbesetzte Vektoren als Lösungen heraus.\\\\
Sei nun $\lambda\geq \|z\|_\infty=\max\limits_{k=1,...,n}|z_k|$. Für $k=1,...,n$ folgt $|z_k|\leq\lambda$ und daher $x_k=0$. Also gilt $x_\lambda=0$.
%Sei $\lambda\geq 0$ beliebig. Dann sind die Richtungsableitungen an der Stelle $x$ in Koordinatenrichtung $e_k,\, k=1,\dots, n$ gegeben durch
%  \begin{displaymath}
%   f'(x,e_k)=(x_k-z_k)+\lambda \sgn(x_k).
%  \end{displaymath}
%  Für die Erfüllung der Variationsungleichung fordern wir weiter
%  \begin{displaymath}
%   0\leq   f'(x,e_k)=(x_k-z_k)+\lambda \sgn(x_k)
%  \end{displaymath}


\end{compactenum}

