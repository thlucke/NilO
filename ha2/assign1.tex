\assignment{2.1}
Es sei $f:\R^n\to \R$ konvex und $x\in \R^n$. Für $h\in \R^n$ bezeichne $f'(x,h)$ die Richtungsableitung von $f$ im Punkt $x$ in Richtung $h$ und
weiter definieren wir 
\begin{displaymath}
 g_h:\R\to\R,\quad g_h(t)=f(x+th).
\end{displaymath}

\begin{compactenum}[(i)]
 \item \footnote{vgl. \url{http://www.mathematik.uni-dortmund.de/~tdohnal/TEACH/Seminar\_AnaIII\_SS2013/Strickmann\_Konvexe\_Fkt.pdf}; S.4}
 Sei $h\in\R^n$ beliebig aber fest gewählt. Mit der Konvexität von $f$ folgt auch die Konvexität von $g_h$;
 betrachte dazu für $t_1,t_2\in \R, \, \lambda\in[0,1]$
 \begin{align*}
  &g(\lambda t_1+(1-\lambda)t_2)&=& f((\lambda+(1-\lambda))x+(\lambda t_1+(1-\lambda)t_2)h)\\
  &&=& f(\lambda(x+t_1h)+(1-\lambda)(x+t_2h))\\
  &&\leq& \lambda f(x+t_1h)+(1-\lambda)f(x+t_2h).
 \end{align*}
 Dann gilt für $t,t_1,t_2\in \R$ mit $t_1<t<t_2$ wie aus Analysis I bekannt
 \begin{displaymath}
  \frac{g(t)-g(t_1)}{t-t_1}\leq\frac{g(t_2)-g(t_1)}{t_2-t_1}\leq\frac{g(t_2)-g(t)}{t_2-t}.
 \end{displaymath}
 Diese Beziehung zeigt, dass der Differenzenquotient monoton wachsend ist: Fixiere dazu jeweils $t_1,t_2$ beziehungsweise $t$.
 Damit ist insbesondere $\frac{g(t)-g(0)}{t}$ für $t\searrow0$ monoton fallend und durch $\frac{g(0)-g(-1)}{-1}$ nach unten beschränkt.
 Somit existiert 
 \begin{displaymath}
  \lim_{t\to 0} \frac{g(t)-g(0)}{t}=\lim_{t\to 0}\frac{f(x+th)-f(x)}{t}=f'(x,h).
 \end{displaymath}
 \item Sei $h\in\R^n$ und $\lambda>0$, dann gilt mit $s=\lambda t$
 \begin{align*}
  &f'(x,\lambda h)&=&\lim_{t\to 0}\frac{f(x+t\lambda h)-f(x)}{t}=\lambda\lim_{t\to 0}\frac{f(x+t\lambda h)-f(x)}{\lambda t}\\
  &&=&\lambda\lim_{s\to 0}\frac{f(x+s h)-f(x)}{s}=\lambda f'(x,h).
 \end{align*}

 \item
 \item
\end{compactenum}


