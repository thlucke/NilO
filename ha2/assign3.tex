\assignment{2.3}
Sei $A\colon \R^n\to\R^m$ linear und $b\in\R^m$. Wir betrachten die Abbildung $f\colon \R^n\to\R$ mit
\begin{displaymath}
 f(x)=\frac{1}{2}\Vert Ax-b\Vert^2=\frac{1}{2}\langle Ax-b,Ax-b\rangle=\frac{1}{2}(\langle A^TAx,x\rangle-2\langle A^Tx,b\rangle+\langle b,b\rangle).
\end{displaymath}
\begin{compactenum}[(i)]
 \item Dann ist die Ableitung in $x\in\R^n$, wie aus Analysis II bekannt, gegeben durch
 \begin{displaymath}
 f'(x)=  A^TAx- A^Tb
 \end{displaymath}
 gegeben. Damit ergibt sich der Gradient
 \begin{displaymath}
  \nabla f(x) = f'(x)^T= x^TA^TA-b^TA^T.
 \end{displaymath}
 \item Ist $\bar x\in\R^n$ ein Minimierer von $f$, dann ist insbesondere die notwendige Optimalitätsbedingung erfüllt, d.h. es gilt $f'(\bar x)=0$
 und somit $A^TA\bar x=A^Tb$. Wie aus Numerik bekannt, ist $A^TA$ eine symmetrische, positiv semidefinite Matrix. Wegen $f''(x)=A^TA$ für alle $x\in\R^n$ ist $f$ also konvex.
 Damit ist die Erfüllung der notwendigen Optimalitätsbedingung bereits hinreichend für Minimalität in einem Punkt.\\
 \item Nach (i) genügt es zu zeigen, dass $A^Tb\in\im A^TA$ ist. Wir zeigen sogar $\im A^T = \im A^TA$. Dabei ist die Inklusion $\im A^T \supset \im A^TA$ klar. Für $x\in\R^n$
 gilt weiterhin
 \begin{displaymath}
  \langle A^TAx,x\rangle=\langle Ax, Ax\rangle
 \end{displaymath}
Damit folgt aus $x\in\ker A^TA$, d.h. $A^TAx=0$, dass
$$\|Ax\|^2=\langle Ax,Ax \rangle=\langle A^TAx,x \rangle=\langle 0,x \rangle=0.$$
Also ist $\|Ax\|=0$ und wegen der positiven Definitheit $Ax=0$, d.h. $x\in\text{ker}(A)$. Umgekehrt folgt aus $x\in\ker A$, d.h. $Ax=0$, dass $A^TAx=0$, d.h. $x\in\ker A^TA$. Also ist $\ker A^TA=\ker A$ und $  \rank A^T A=\rank A=\rank A^T$.  Daher muss bereits $\im A^T = \im A^TA$ gelten, d.h. es gibt eine Lösung $\bar x\in\R^n$ von $A^TAx=A^Tb$.\\
 \item Die quadratische Matrix $A^TA\in \R^{n,n}$ hat wegen (iii) vollen Rang $\rank A^TA=\rank A=n$ und ist somit invertierbar, d.h. die zugehörige Abbildung $\R^n\to\R^n$ ist bijektiv. Folglich erfüllt nur $x=(A^TA)^{-1}A^Tb$ die notwendige Optimalitätsbedingung und ist nach (ii) einziger Minimierer.
\end{compactenum}
