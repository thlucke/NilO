\assignment{6.4}

Seien $\mathcal F:=\left\{\begin{bmatrix}x\\y\end{bmatrix}\in\mathbb R^2\mid \forall j\in\{1,2,3\}:g_j(x,y)\leq 0\right\}$ mit 
\begin{enumerate}
\item $g_1: \mathbb R^2\to\mathbb R$ mit $g_1(x,y)=(x-1)^2+(y-1)^2-2$
\item $g_2: \mathbb R^2\to\mathbb R$ mit $g_2(x,y)=(x-1)^2+(y+1)^2-2$
\item $g_3: \mathbb R^2\to\mathbb R$ mit $g_3(x,y)=-x$
\end{enumerate}
Die zugehörigen Gradienten sind
\begin{enumerate}
\item $\nabla g_1(x,y)=\begin{bmatrix}2(x-1)\\2(y-1)\end{bmatrix}$.
\item $\nabla g_2(x,y)=\begin{bmatrix}2(x-1)\\2(y+1)\end{bmatrix}$.
\item $\nabla g_3(x,y)=\begin{bmatrix}-1\\0\end{bmatrix}$
\end{enumerate}
Wir betrachten $\begin{bmatrix}x\\y\end{bmatrix}=\begin{bmatrix}0\\0\end{bmatrix}$ und untersuchen folgende Regularitätsbedingungen:
\begin{compactenum}[(i)]
\item Bedingung der linearen Unabhängigkeit (LICQ):\\
In $\begin{bmatrix}0\\0\end{bmatrix}$ sind alle Ungleichungen aktiv. Die drei je zweikomponentigen Gradienten $\nabla g_1(0,0),\nabla g_2(0,0),\nabla g_3(0,0)\in\mathbb R^2$ können zusammen nicht linear unabhängig sein. Daher ist (LICQ) nicht erfüllt.\\
\item Bedingung von Mangasarian und Fromovitz (MFCQ):\\
Mit $d=\begin{bmatrix}1\\0\end{bmatrix}\in\mathbb R^2$ gilt
\begin{enumerate}
\item $\left\langle \nabla g_1(0,0),d\right\rangle=\left\langle \begin{bmatrix}-2\\-2\end{bmatrix},\begin{bmatrix}1\\0\end{bmatrix}\right\rangle=-2<0$
\item $\left\langle \nabla g_2(0,0),d\right\rangle=\left\langle \begin{bmatrix}-2\\2\end{bmatrix},\begin{bmatrix}1\\0\end{bmatrix}\right\rangle=-2<0$
\item $\left\langle \nabla g_3(0,0),d\right\rangle=\left\langle \begin{bmatrix}-1\\0\end{bmatrix},\begin{bmatrix}1\\0\end{bmatrix}\right\rangle=-1<0$
\end{enumerate}
Da die Bedingungen an Gleichungsrestriktionen wegfallen, ist (MFCQ) erfüllt.
\end{compactenum}
