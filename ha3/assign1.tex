\assignment{3.1}
Obwohl die in der Aufgabenstellung formulierte vermeintliche Charakterisierung
der Abstiegsrichtung au"serordentlich intuitiv erscheint, ist sie im Allgemeinen
nicht mit dem in der Vorlesung eingef"uhrten Begriff der Abstiegsrichtung vereinbar.
Das folgende Gegenbeispiel dient als Untermauerung dieser k"uhnen Behauptung.\\

F"ur $n=1$ bem"uhen wir die Sinusfunktion $\sin\colon\R\to\R, x\mapsto\sin(x)$
und betrachten den Punkt $x = \pi /2$. Dann ist mit $d=1$ wegen
\[
\nabla \sin(\pi/2)\cdot d = \cos(\pi/2) \cdot 1 = 0 \nless 0
\]
keine Abstiegsrichtung im Sinne der Definition (4.1.1) gegeben. Setzen wir allerdings
$\widetilde{\sigma} := \pi/2$, so ist f"ur alle $\sigma\in (0,\widetilde{\sigma}]$
die geforderte Absch"atzung
\[
0 \le \sin(x + \sigma d) < 1 = \sin(x)
\]
dennoch erf"ullt. Es kann also keine "Aquivalenz gelten.
