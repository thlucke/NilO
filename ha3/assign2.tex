\assignment{3.2}

Es sei $f\colon\R^n\to\R$ stetig differenzierbar, streng konvex und koerziv.
Damit hat $f$ einen eindeutigen Minimierer $\bar x\in\R^n$ und  es gilt genau dann $\nabla f(x)=0$,
wenn $x=\bar x$ ist. ($\dagger$)\\ 
Sei $x^0\in\R^n$ beliebig.
Die im Abstiegsverfahren gewählte Richtung $x\mapsto d(x)$ genüge
folgender Bedingung: Es gibt Konstanten $c_1,c_2>0$ und $p>-1$ mit
\begin{displaymath}\label{star}
 \frac{|\langle \nabla f(x),d(x)\rangle|}{\Vert\nabla f(x)\Vert\Vert d(x)\Vert}\geq\min\{c_1,c_2\Vert \nabla f(x)\Vert^p\}\tag{$\star$}
\end{displaymath}
für alle $x\in\R^n$. Weiter seien die Schrittweiten $(x,d(x))\mapsto \sigma(x)$
effizient gewählt, d.h. es gibt ein $c_3>0$ mit
\begin{displaymath}\label{doublestar}
 f(x+d(x)\sigma(x))\leq f(x)-c_3\left(\frac{\langle\nabla f(x),d(x)\rangle}{\Vert d(x)\Vert}\right)^2 \tag{$\star\star$}
\end{displaymath}
für alle $x\in N(f,f(x^0))$.
\\ \\
Wir zeigen zunächst, dass die Folge der Gradienten
$(\nabla f(x^k))_{k\in\N}$ des  Verfahrens gegen Null konvergiert.
\\\\
Wegen
\begin{displaymath}
 c_3\left(\frac{\langle\nabla f(x^k),d(x^k)\rangle}{\Vert d(x^k)\Vert}\right)^2\geq0
\end{displaymath}
und  mit \eqref{doublestar} ist die Folge $(f(x^k))_{k\in\N}$ monoton fallend und aufgrund der Beschränktheit von $f$ auch konvergent.
Von \eqref{doublestar} ausgehend, erhalten wir zudem für $k\in\mathbb{N}$
\begin{displaymath}
 c_3\left(\frac{\langle\nabla f(x^k),d(x^k)\rangle}{\Vert d(x^k)\Vert}\right)^2 \leq f(x^k)-f(x^{k+1}) \overset{k\to\infty}{\to} 0.
\end{displaymath}
Setzen wir nun \eqref{star} ein, so ergibt sich
\begin{align*}
 c_3\Vert\nabla f(x^k)\Vert^2 (\min\{c_1,c_2\Vert \nabla f(x^k)\Vert^p\})^2 &\leq c_3\left(\frac{\langle\nabla f(x^k),d(x^k)\rangle}{\Vert d(x^k)\Vert}\right)^2 \\
 &\leq f(x^k)-f(x^{k+1}) \overset{k\to\infty}{\to} 0.
\end{align*}
Damit gilt $c_3c_1^2\Vert\nabla f(x^k)\Vert^2\to 0$ oder $c_3c_2^2\Vert\nabla f(x^k)\Vert^{2+2p}\to 0$ für $k\to\infty$,
was genau dann erfüllt ist, falls $(\nabla f(x^k))_k$ gegen Null konvergiert.
\\\\
Aus der Koerzivität und der Stetigkeit von $f$ folgt, dass die Niveaumenge $N(f,f(x^0))$ kompakt ist.
Damit hat die jede Teilfolge der Folge $(x^k)_{k\in\N}$ eine konvergente Teilfolge $(x^{k_{l_m}})_{m\in\N}$, deren Grenzwert wir mit $\tilde x_l\in\R^n$
bezeichnen wollen. Mit der Stetigkeit der Ableitung von $f$ folgt
\begin{displaymath}
 \nabla f(\tilde x_l)=\lim_{m\to\infty}\nabla f(x^{k_{l_m}})=0.
\end{displaymath}
Mit ($\dagger$) folgt, dass $\tilde x_l$ der (eindeutige) Minimierer $\bar x$ von $f$ ist. Insgesamt konvergiert also
eine Teilfolge jeder Teilfolge gegen $\bar x$ und somit konvergiert die ganze Folge.
