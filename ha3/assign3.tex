\assignment{3.3}

Gegeben sei eine stetig differenzierbare Funktion $f\colon R^n\to \R$
und ein Startwert $x^0\in\R^n$.\\
\begin{compactenum}[(i)]
 \item Es besitze $f$ einen Minimierer; insbesondere ist $f$ dann beschränkt.
 Weiter sei $(x^k)_{k\in\N}$ eine durch ein Abstiegsverfahren mit effizienter
 Schrittweite erzeugte Folge, d.h. es gibt ein $c>0$ mit
\begin{displaymath}
 f(x^{k+1})\leq f(x^k)-c\left(\frac{\langle\nabla f(x^k),d(x^k)\rangle}{\Vert d(x^k)\Vert}\right)^2 \quad\text{für alle }k\in\N.
\end{displaymath}
 Damit ist die Folge $(f(x^k))_{k\in\N}$ monoton fallend und beschränkt, weil
 $f$ beschränkt ist. Somit ist $(f(x^k))_{k\in\N}$ konvergent. Durch Kontraposition
 folgt die Behauptung: Ist $(f(x^k))_{k\in\N}$ divergent, so hat $f$ keinen Minimierer.
 \\
 \item Sei $(x^k)_{k\in\N}$ eine durch ein Abstiegsverfahren mit effizienter Schrittweite erzeugte
 Folge und sei $x$ ein Häufungspunkt von $(x^k)_{k\in\N}$, sowie $(x^{k_l})_{l\in\N}$ die gegen $x$ konvergente
 Teilfolge. Weil die Schrittweite effizient gewählt ist und für ein Abstiegsverfahren
  stets $\langle\nabla f(x),d(x)\rangle<0$ gilt, ist wegen
 \begin{displaymath}
 f(x^{k+1})\leq f(x^k)-c\left(\frac{\langle\nabla f(x^k),d(x^k)\rangle}{\Vert d(x^k)\Vert}\right)^2 \quad\text{für alle }k\in\N, \, c>0
\end{displaymath}
 die Bildfolge $(f(x^k))_{k\in\N}$ \textit{streng} monoton fallend. Insbesondere konvergiert dann wegen der Stetigkeit von $f$
 auch $(f(x^{k_l}))_{l\in\N}$ streng monoton von \textit{oben} gegen $f(x)$, d.h. für jedes $\varepsilon >0$ gibt es ein
 $l\in\mathbb{N}$, sodass $\Vert x^{k_l}-x\Vert <\varepsilon$ und $f(x^{k_l})>f(x)$ ist. Somit kann $x$ kein lokaler Maximierer sein.
 \\
 \item Wir zeigen, dass diese Aussage nicht wahr sein kann. Wir betrachten dazu für $n=1$
 die Funktion $f=\sin(\sqrt{\frac{3\pi}{2}}\cdot)$ und den Startwert $x_0=0$. Dann ist
 \begin{displaymath}
  f(x^1)=f(x_0-\nabla f(x^0))=\sin\left(\sqrt{\frac{3\pi}{2}}\left(0-\sqrt{\frac{3\pi}{2}}\right)\right)=1>0=\sin(x^0)=f(x).
 \end{displaymath}
 Damit gilt insbesondere für alle $c>0$
 \begin{displaymath}
 f(x_1)> f(x^0)-c\left(\frac{\langle\nabla f(x_0),\nabla f(x_0)\rangle}{\Vert \nabla f(x_0)\Vert}\right)^2,
\end{displaymath}
d.h. $\sigma=1$ ist keine effiziente Schrittweite.
\\
 \item Auch diese Aussage trifft nicht zu. Wir untersuchen im Fall $n=1$ die Funktion $f\colon x\mapsto x^3$ und $x_0=-1$.
 Dann ist $x_1=-1-1\cdot 2(-1)^2=-3$ und es gilt
 \begin{displaymath}
  |f(x_0)|=|2(-1)^2|<|2(-3)^2|=|f(x_1)|.
 \end{displaymath}

\end{compactenum}
