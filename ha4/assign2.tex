\assignment{4.2}

Es sei $K\subset \R^d$ ein konvexer Kegel mit $0\in K$.
\\
\begin{compactenum}[(i)]
 \item Wir zeigen, dass $K^*$ abgschlossen ist. Sei dazu $(x_n)_n\subset K^*$ eine in $\R^d$ konvergente
 Folge mit Grenzwert $x\in\R^d$. Sei $y\in K$ beliebig. Nach Definition des Dualkegels gilt 
 \begin{displaymath}
  \langle x_n,y\rangle\leq 0
 \end{displaymath}
 und mit der Stetigkeit der dualen Paarung folgt
 \begin{displaymath}
  \langle x, y\rangle =\lim_{n\to\infty}\langle x_n,y\rangle\leq 0.
 \end{displaymath} 
 Weil $y$ beliebig war, ist $x\in K^*$.
 \\
 \item Wir zeigen, dass $K$ genau dann abgeschlossen ist, wenn $K=K^{**}$ ist.
 \begin{compactenum}
 \item[$\Leftarrow$] Wegen $K^{**}=(K^*)^*$ folgt mit (i) die Abgeschlossenheit von $K^{**}=K$.
 \item[$\Rightarrow$] Für jeden Kegel gilt $K\subset K^{**}$, denn ist $x\in K$ beliebig,
 so gilt für alle $s\in K^*$
 \begin{displaymath}
    \langle x,s\rangle\leq 0,
 \end{displaymath}
 d.h. $x\in K^{**}$. Bleibt noch die zweite Inklusion $K\supset K^{**}$ zu zeigen. Angenommen, es gäbe
 $y\in K^{**}\backslash\{K\}$. Nach dem Trennungssatz gibt es ein $s\in \R^d$ mit 
 \begin{displaymath}
  \langle s, y-x\rangle > 0
 \end{displaymath}
 für alle $x\in K$. Isbesondere ist also
 \begin{displaymath}
  \langle s,y\rangle >0.
 \end{displaymath}
 Für $\lambda>0$ und $x\in K$ gilt 
 \begin{displaymath}
  0<\langle s, y-\lambda x\rangle=\langle s, y\rangle-\lambda\langle s, x\rangle.
 \end{displaymath}
 Weil $\lambda$ beliebig war, folgt $\langle s, x\rangle\leq 0$ für alle $x\in K$. 
 Damit ist $s\in K^*$. Dann müsste aber auch
 \begin{displaymath}
  \langle s,y\rangle \leq 0
 \end{displaymath}
 sein, was uns den ersehnten Widerspruch liefert. Folglich ist $K\supset K^{**}$.
 \\
 \end{compactenum}
 \item  Es sei $f\in K^*$ und es gelte $\langle f, x_0\rangle\leq 0$ für einen einen inneren Punkt $x_0\in K$.
 Wir zeigen mittels Widerspruchsbeweis, dass $f=0$ ist. Sei also $f\neq0$. Da $x_0$ ein innerer Punkt von $K$ 
 ist, gibt es ein $\varepsilon>0$ mit $U_\varepsilon(x_0)\subset K$. Damit ist insbesondere 
 $x_0+\frac{\varepsilon}{\Vert f \Vert} f\in K$ und es gilt
 \begin{displaymath}
  0\geq \langle f,x_0+\frac{\varepsilon}{\Vert f \Vert} f\rangle
  = \langle f,x_0\rangle+\langle f,\frac{\varepsilon}{\Vert f \Vert} f\rangle
  =0+\varepsilon \Vert f\Vert.
 \end{displaymath}
 Dies kann aber nur gelten, wenn $\Vert f\Vert=0$ ist, was im Widerspruch zur Annahme $f\neq 0$ steht.
 \\
 \item Diese Aussage ist falsch. Dazu betrachten wir für $d=2$ den Kegel
 \begin{displaymath}
  K\coloneqq\left\{\begin{pmatrix}x_1\\x_2\end{pmatrix}: 0< x_1 \land 0< x_2 \right\}\cup\{0\}
 \end{displaymath}
  und
  \begin{displaymath}
  K^*=\left\{\begin{pmatrix}s_1\\s_2\end{pmatrix}: 0\geq s_1 \land 0\geq s_2 \right\}.
 \end{displaymath}
 Dass $K$ und $K^*$ Kegel sind, ist offensichtlich. Außerdem gilt für jedes $s\in K^*$
 \begin{displaymath}
  \langle s,x\rangle =s_1x_1+s_2x_2\leq 0
 \end{displaymath}
 für alle $x\in K$. Andererseits ist für $y\in \R^2\backslash K^*$ entweder $y_1>0$ oder $y_2>0$;
 sei o.B.d.A. $y_1>0$. Dann gilt für $x\coloneqq(y_1+|y_2|,\frac{y_1}{2})^T\in K$
 \begin{displaymath}
  \langle x, y\rangle \geq y_1^2+\frac{|y_2|y_1}{2}>0.
 \end{displaymath}
 Damit ist gezeigt, dass $K^*$ tatsächlich der Dualkegel von $K$ ist. Offensichtlich ist der 
 erste Einheitsvektor nicht in $K$ enthalten, dennoch gilt
  \begin{displaymath}
  \langle e_1, s\rangle \leq 0
 \end{displaymath}
 für alle $s\in K^*$.
\end{compactenum}
