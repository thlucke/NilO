\assignment{4.5}

Sei $A\colon \R^n\to\R^n$ symmetrisch und positiv definit (und damit invertierbar) sowie $b\in\R^n$.
Ferner bezeichne $\Vert\cdot\Vert_A$ die von $A$ induzierte Norm und
$\Vert\cdot\Vert_{A^{-1}}$ die von $A^{-1}$ induzierte Norm. Sei $x\in\R^n$
die eindeutige Lösung von $Ax=b$ und sei $\tilde x\in\R^n$ ein beliebiger Vektor.
\\ \\
Dann gilt
\begin{align*}
 \Vert x-\tilde x \Vert_A^2 &=(x-\tilde x)^TA(x-\tilde x)\\&=(A^{-1}b-\tilde x)^T(b-A\tilde x)\\
 =&(A^{-1}(b-A\tilde x))^T(b-A\tilde x)\\&=(b-A\tilde x)^T(A^{-1})^T(b-A\tilde x)\\&=(b-A\tilde x)^TA^{-1}(b-A\tilde x)\\&=\Vert b-A\tilde x\Vert_{A^{-1}}^2.
\end{align*}
Mit der Dreiecksungleichung folgt dann für beliebiges $w\in\R^n$
\begin{align*}
\Vert x-\tilde x \Vert_A&=\Vert b-(w-w)-A\tilde x\Vert_{A^{-1}}\\
&=\Vert (b-w)+(w-A\tilde x)\Vert_{A^{-1}}\\
&\leq\Vert b-w\Vert_{A^{-1}}+\Vert w-A\tilde x\Vert_{A^{-1}}\\
&=\Vert b-w\Vert_{A^{-1}}+\Vert A\tilde x-w\Vert_{A^{-1}}.
\end{align*}
Wir betrachten nun $n=1$, $A=\id$, $b=0$ (und somit $x=\id^{-1}(b)=0$), $\tilde x=1$ und $w=0$. Dann gilt die Gleichheit
\begin{displaymath}
 |x-\tilde x|=|0-1|=1=|0-0|+|1-0|=|b-w|+|A\tilde x-w|.
\end{displaymath}
