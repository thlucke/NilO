\assignment{4.5}

Es sei $A\colon \R^n\to\R^n$ symmetrisch und positiv de nit sowie $b\in\R^n$.
Ferner bezeichne $\Vert\cdot\Vert_A$ die von $A$ induzierte Norm und
$\Vert\cdot\Vert_{A^{-1}}$ die von $A^{-1}$ induzierte Norm. Es sei $x\in\R^n$
die Lösung von $Ax=b$ und sei $\tilde x\in\R^n$ ein völlig beliebiger Vektor.
\\ \\
Dann gilt
\begin{align*}
 &\Vert x-\tilde x \Vert_A^2 &=&(x-\tilde x)^TA(x-\tilde x)\\&&=&(A^{-1}b-\tilde x)^T(b-A\tilde x)\\
 &&=&(A^{-1}(b-A\tilde x))^T(b-A\tilde x)\\&&=&\Vert b-A\tilde x\Vert_{A^{-1}}^2.
\end{align*}
Mit der Dreiecksungleichung folgt dann für beliebiges $w\in\R^n$
\begin{displaymath}
 \Vert x-\tilde x \Vert_A=\Vert b-(w-w)-A\tilde x\Vert_{A^{-1}}\leq \Vert b-w\Vert_{A^{-1}}+\Vert A\tilde x-w\Vert_{A^{-1}}.
\end{displaymath}

Wir betrachten nun $n=1$ und $A=\id$. Dann gilt z.B. für $x=0$, $\tilde x=1$ und $w=0$ die Gleichheit
\begin{displaymath}
 |x-\tilde x|=|0-1|=1=|0-0|+|1-0|=|b-w|+|A\tilde x-w|.
\end{displaymath}
