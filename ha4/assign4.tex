\assignment{4.4}
\newcommand{\e}{\varepsilon}
\newcommand{\M}{\partial{M}}

Bevor wir die gestellten Aufgaben l"osen, beweisen wir den folgenden Hilfssatz.\\

\textbf{Lemma.} Ist $S:=\{x\in\R^2 \mid \|x\|_2 \le 1\}$, so gilt $K(S,0)=\R^2$.

\begin{proof}
Sei $x\in\R^2$ vorgegeben. Dann k"onnen wir dieses $x$ durch die Wahl eines
geeigneten $\varphi_x \in [0,2\pi[$ in Polarkoordinaten wie folgt darstellen:
\[
x = \|x\|_2 \cdot (\cos(\varphi_x), \sin(\varphi_x)).
\]
Setzen wir $\widetilde{x} := (\cos(\varphi_x), \sin(\varphi_x))$, so ist $\|\widetilde{x}\|_2 = 1$
und damit $\widetilde{x} \in S$. Nach Konstruktion folgt dann aus
\[
x = \|x\|_2\cdot(\widetilde{x}-0) \in K(S,0)
\]
die Behauptung.
\end{proof}

Nun widmen wir uns den gestellten Aufgaben. Dabei "ubernehmen wir jeweils die
gegebenen Voraussetzungen und Notationen.

\begin{itemize}
\item [(i)]
Da die Eigenschaft der Konvexit"at translationsinvariant
ist, k"onnen wir ohne Einschr"ankung annehmen, dass $x_0 = 0$ gilt. Da $x_0$
nach Voraussetzung ein innerer Punkt von $C$ ist, muss ein $\e > 0$ existieren,
mit
\[
\overline{U_\e (0)} \subseteq C.
\]
Da Konvexit"at auch unter Skalierung invariant bleibt, k"onnen wir ohne Beschr"ankung
der Allgemeinheit $\e = 1$ annehmen.\\

Ist nun ein $x\in\R^n$ vorgegeben, dann gibt es einen Unterraum $E_x$ mit $\dim(E)=2$,
so dass $x \in E_x$ und trivialerweise auch $x_0 \in E$ gilt. Da $E_x$ isomorph zu $\R^2$
ist und wir $E_x \cap \overline{U_\e (0)}$ mit $S$ identifizieren k"onnen,
folgt aus dem eingangs bewiesenen Lemma $K(C,0)\cap E_x = E_x$. Da $x$ beliebig
gew"ahlt war, gilt folglich
\[
K(C,0) = \bigcup_{x\in\R^n} E_x = \R^n.
\]
Des Weiteren gilt $N(C,0) = \{0\}$. W"are dies nicht der Fall, so m"usste eine
Normalenrichtung $s\in\R^n\setminus\{0\}$ existieren, die f"ur alle $y\in C$ die Absch"atzung
\[
\langle s, y \rangle \le 0
\]
erf"ullt. Wegen $C = \R^n$ muss diese Bedingung insbesondere f"ur $y = s$ erf"ullt
sein. Das ist aber der positiven Definitheit von Skalarprodukten wegen nicht m"oglich.
Es gilt n"amlich $\langle s,s \rangle > 0$.

\item[(ii)] Es sei $H^+ := \{(x_1, x_2)^T \in\R^2 \mid x_2 > 0\}$ die offene obere Halbebene
des $\R^2$. Wir werden die Mengengleichheit $K(M,0) = H^+$ beweisen.
\begin{proof}
Zun"achst bemerken wir, dass $K(H^+, 0) = H^+$ gilt. W"are dem nicht so, dann
lie"se sich ein $y = (y_1, y_2)^T \in \R^2$ mit $y_2 \le 0$ und $y\in K(H^+,0)$
finden. Wegen $K(H^+,0) = \{(\alpha x_1, \alpha x_2)^T \in \R^2 \mid \alpha > 0, x_2 > 0\}$ kann aber
so ein $y$ nicht existieren. Wegen $M\subseteq H^+$ folgt dann unmittelbar
$K(M, 0) \subseteq K(H^+, 0) = H^+$.\\

Nun zur anderen Mengeninklusion: Betrachten wir zun"achst den Rand $\M$
von $\overline{M}$. Dann ist die eindeutig bestimmte Tangente von $\M$ an $x_0$ durch den
Unterraum $\text{Span}{(1,0)^T}$ gegeben. Ist nun ein $x=(x_1, x_2)\in H^+$ gegeben,
dann folgt aus $x_2 > 0$ f"ur den Winkel $\measuredangle((1,0),x)$ zwischen $x$
und dem ersten Einheitsvektor
\[
\measuredangle((1,0),x) \in ]0,\pi[.
\]
Demnach
muss die Verbindungsstrecke $\overline {x_0 x}$ oberhalb der Tangente liegen.
Da nun aber $x_0 \in \M$ gilt, hat $\overline {x_0 x}$ zwei Schnittpunkte mit $\M$. Es gibt
daher ein von $x_0$ und $x$ verschiedenes Element $y \in \overline {x_0 x}$ mit
$y\in M$. Wegen
\[
x = \frac{\|x\|}{\|y\|}\cdot(y - 0)
\]
folgt somit aus $x\in K(H^+,0)$ die gew"unschte Gleichheit.
\end{proof}

Zum Abschluss zeigen wir noch die Mengengleichheit
\[
N':= \{(0,x_2) \in \R^2 \mid x_2 \le 0\} = N(M,0).
\]
Sei daf"ur zun"achst ein $s = (0,s_2) \in N'$ gegeben. Dann gilt f"ur alle
$y = (y_1, y_2) \in H^+$ die Identit"at
$
\langle s,y-0 \rangle = s_2 y_2
$.
Wegen $s_2 \le 0$ und $y_2 > 0$, folgt $s_2 y_2 \le 0$ und damit $s \in N(M,0)$.\\

Sei nun $s=(s_1, s_2) \in N(M,0)$. Dann gilt nach Definition f"ur alle $y = (y_1, y_2)\in M$
\begin{equation}\label{blabla}
\langle s,y-0 \rangle = s_1 y_1 + s_2 y_2 \le 0.
\end{equation}
%Ist $y = (y_1,0)$ f"ur ein $y_1 \in \R$, so folgt $s_1 y_1 + s_2 y_2= s_1 y_1$. Um \eqref{blabla}
%f"ur alle $y_1 \in \R$ erf"ullen zu k"onnen, folgt daher $s_1 = 0$. Andernfalls
%ist mit $y_1 = s_1$ stets $s_1 y_1 > 0$.
Ist $y = (0, y_2)$ f"ur ein $y_2 \in \R$, so folgt mit \eqref{blabla}
gerade $s_1 y_1 + s_2 y_2= s_2 y_2$. Wegen $y_2 > 0$ kann die geforderte Ungleichung
nur im Falle $s_2 \le 0$ erf"ullt werden. Ist $s_1 \neq 0$, dann folgt mit $y=(-s_2/s_1 + 1, 1)$
\[
s_1 y_1 + s_2 y_2 = 1 > 0.
\]
Demnach muss $s_1 = 0$ gelten. Damit folgt dann endlich $N(M,0)= N'$. \hfill $\Box$

\item[(iii)] Es sei $f\colon\R^3 \to \R, f(x) =\frac{1}{2} (x_1^2 + x_2^2 + x_3^2)$ und
$g\colon\R^3 \to \R, g(x) = x_1 + x_2 + x_3 - 6$. Dann ist die Lagrange-Funktion
f"ur das in der Aufgabe gestellte Minimierungsproblem durch
\[
L(x, \lambda) = f(x) + \lambda g(x)
\]
gegeben. Wir untersuchen die notwendige Optimalit"atsbedingung
\[
\begin{pmatrix} \nabla_x L(x, \lambda) \\ \nabla_\lambda L(x, \lambda)\end{pmatrix}
= \begin{pmatrix}
x_1+\lambda\\x_2+\lambda\\x_3+\lambda \\ x_1 + x_2 + x_3 - 6 \end{pmatrix}
\overset{!}{=}\begin{pmatrix} 0 \\ 0\\0\\0 \end{pmatrix}.
\]
Die Erf"ullung der Optimalit"atsbedingung
ist damit "aquivalent zur Existenz einer L"osung des linearen Gleichungssystems
\[
\begin{pmatrix}
1 & 0 & 0 &1 \\
0 & 1 & 0 &1 \\
0 & 0 & 1 &1\\
1 & 1 & 1 & 0
\end{pmatrix}\begin{pmatrix} x_1 \\ x_2 \\ x_3 \\\lambda\end{pmatrix}
= \begin{pmatrix} 0\\0\\0 \\ 6 \end{pmatrix}.
\]
Dieses System besitzt die eindeutige Lösung
\[\begin{pmatrix} x_1 \\ x_2 \\ x_3 \\\lambda\end{pmatrix}
=\begin{pmatrix} 2 \\ 2 \\ 2 \\-2\end{pmatrix},
\]
d.h. bei $x=(2,2,2)^T$ wird das Minimum von $f(x)=6$ angenommen.
\end{itemize}
