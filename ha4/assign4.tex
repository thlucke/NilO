\assignment{4.4}
\newcommand{\e}{\varepsilon}

Bevor wir die gestellten Aufgaben l"osen, beweisen wir den folgenden Hilfssatz.\\

\textbf{Lemma.} Ist $S:=\{x\in\R^2 \mid \|x\|_2 \le 1\}$, so gilt $K(S,0)=\R^2$.

\begin{proof}
Sei $x\in\R^2$ vorgegeben. Dann k"onnen wir dieses $x$ durch die Wahl eines
geeigneten $\varphi_x \in [0,2\pi[$ in Polarkoordinaten wie folgt darstellen:
\[
x = \|x\|_2 \cdot (\cos(\varphi_x), \sin(\varphi_x)).
\]
Setzen wir $\widetilde{x} := (\cos(\varphi_x), \sin(\varphi_x))$, so ist $\|\widetilde{x}\|_2 = 1$
und damit $\widetilde{x} \in S$. Nach Konstruktion folgt dann aus
\[
x = \|x\|_2\cdot(\widetilde{x}-0) \in K(S,0)
\]
die Behauptung.
\end{proof}

Nun widmen wir uns den gestellten Aufgaben. Dabei "ubernehmen wir jeweils die
gegebenen Voraussetzungen und Notationen.

\begin{itemize}
\item [(i)]
Da die Eigenschaft der Konvexit"at translationsinvariant
ist, k"onnen wir ohne Einschr"ankung annehmen, dass $x_0 = 0$ gilt. Da $x_0$
nach Voraussetzung ein innerer Punkt von $C$ ist, muss ein $\e > 0$ existieren,
mit
\[
\overline{U_\e (0)} \subseteq C.
\]
Da Konvexit"at auch unter Skalierung invariant bleibt, k"onnen wir ohne Beschr"ankung
der Allgemeinheit $\e = 1$ annehmen.\\

Ist nun ein $x\in\R^n$ vorgegeben, dann gibt es einen Unterraum $E_x$ mit $\dim(E)=2$
sodass $x \in E_x$ und trivialerweise auch $x_0 \in E$ gilt. Da $E_x$ isomorph zu $\R^2$
ist, folgt aus dem eingangs bewiesenen Lemma $K(C,0)\cap E_x = E_x$. Da $x$ beliebig
gew"ahlt war, gilt folglich
\[
K(C,0) = \bigcup_{x\in\R^n} E_x = \R^n.
\]
Des Weiteren gilt $N(C,0) = \{0\}$. W"are dies nicht der Fall, so m"usste eine
Normalenrichtung $s\in\R^n\setminus\{0\}$ existieren, die f"ur alle $y\in C$ die Absch"atzung
\[
\langle s, y \rangle \le 0
\]
erf"ullt. Wegen $C = \R^n$ muss diese Bedinungen insbesondere f"ur $y = s$ erf"ullt
sein. Das ist aber der positiven Definitheit von Skalarprodukten wegen nicht m"oglich.
Es gilt n"amlich $\langle s,s \rangle > 0$.
\end{itemize}
