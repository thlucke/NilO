\assignment{4.3}

Es sei $C\in\R^d$ eine nichtleere, konvexe und abgeschlossene Menge. Für einen Punkt $x_0\in\R^d$ definieren wir
\begin{displaymath}
 f\colon \R^d\to\R,\, f(x)=\frac{1}{2}\Vert x-x_0\Vert^2.
\end{displaymath}
Sei $P_C(x_0)$ die eindeutige Lösung von $\min\limits_{x\in C}\Vert x-x_0\Vert$. Wegen der Monotonie und der Bijektivität von $q: \mathbb R_{\geq 0}\to \mathbb R_{\geq 0}, x\mapsto \frac{1}{2}x^2$ ist $P_C(x_0)$ auch die eindeutige Lösung von $$\min_{x\in C}f(x)=\min_{x\in C}\frac{1}{2}\Vert x-x_0\Vert^2.$$
\begin{compactenum}[(i)]
 \item Seien $x,x_0\in \R^d$. Da $f$ strikt konvex ist, gilt genau dann $p(x_0)=x$, wenn $x$ die Variationsungleichung
 \begin{displaymath}
  0\leq \langle \nabla f(x),y-x\rangle=\langle x-x_0,y-x\rangle \quad\text{ für alle }y\in C
 \end{displaymath} erfüllt.
 Dies ist äquivalent zu
 \begin{displaymath}
  0\geq \langle x_0-x,y-x\rangle \quad\text{ für alle }y\in C,
 \end{displaymath}
 mit anderen Worten $x_0-x\in N(C,x)$.
 \\
 \item Sei zusätzlich $x_1\in \R^d$. Wegen $P_C(x_0),P_C(x_1)\in C$ haben wir mit Aufgabenteil (i)
 \begin{displaymath}
    0\geq \langle x_0-P_C(x_0),P_C(x_1)-P_C(x_0)\rangle 
 \end{displaymath}
 und
  \begin{displaymath}
    0\geq \langle x_1-P_C(x_1),P_C(x_0)-P_C(x_1)\rangle.
 \end{displaymath}
 Durch Addition ergibt sich
 \begin{align*}
 & 0&\geq& \langle x_0-P_C(x_0),P_C(x_1)-P_C(x_0)\rangle +\langle x_1-P_C(x_1),P_C(x_0)-P_C(x_1)\rangle \\
 &&=& \langle x_0-P_C(x_0),P_C(x_1)-P_C(x_0)\rangle +\langle P_C(x_1)-x_1,P_C(x_1)-P_C(x_0)\rangle \\
 &&=&\langle P_C(x_1)-P_C(x_0),P_C(x_1)-P_C(x_0)\rangle+\langle x_0-x_1,P_C(x_1)-P_C(x_0)\rangle\\
&&=&\| P_C(x_1)-P_C(x_0)\|^2+\langle P_C(x_1)-P_C(x_0),x_0-x_1\rangle
 \end{align*}
Daraus folgt mithilfe der Cauchy-Schwarz-Ungleichung
 \begin{displaymath}
  \Vert P_C(x_1)-P_C(x_0)\Vert^2 \leq \langle x_1-x_0,P_C(x_1)-P_C(x_0)\rangle\leq \Vert x_1-x_0\Vert \Vert P_C(x_1)-P_C(x_0)\Vert
 \end{displaymath}
 und weiter
 \begin{displaymath}
  \Vert P_C(x_1)-P_C(x_0)\Vert\leq \Vert x_1-x_0\Vert,
 \end{displaymath}
 also die Lipschitz-Stetigkeit und schließlich die Stetigkeit von $P_C$.
\end{compactenum}


