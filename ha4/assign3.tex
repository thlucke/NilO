\assignment{4.3}

Es sei $C\in\R^d$ eine nichtleere, konvexe und abgeschlossene Menge. Für einen Punkt $x_0\in\R^d$ definieren wir
\begin{displaymath}
 f\colon \R^d\to\R,\, f(x)=\frac{1}{2}\Vert x-x_0\Vert^2
\end{displaymath}
und
\begin{displaymath}
 P_C(x_0)\coloneqq \arg\min_{x\in C}f(x)=\arg\min_{x\in C}\Vert x-x_0\Vert.
\end{displaymath}
\begin{compactenum}[(i)]
 \item Seien $x,x_0\in \R^d$. Da $f$ strikt konvex ist, ist die Lösung des Problems eindeutig.
 Und es gilt genau dann $p(x_0)=x$, wenn $x$ die Variationsungleichung
 \begin{displaymath}
  0\leq \langle \nabla f(x),y-x\rangle=\langle x-x_0,y-x\rangle \quad\text{ für alle }y\in C
 \end{displaymath} erfüllt.
 Dies ist äquivalent zu
 \begin{displaymath}
  0\geq \langle x_0-x,y-x\rangle \quad\text{ für alle }y\in C,
 \end{displaymath}
 mit anderen Worten $x_0-x\in N(C,x)$.
 \\
 \item Wir untersuchen
 \begin{displaymath}
  \Vert P_C(x_0)-x_0\Vert=\Vert (\arg\min_{x\in C}\Vert x-x_0\Vert)-x_0\Vert =\min_{x\in C}\Vert x-x_0\Vert,
 \end{displaymath}
 was gerade dem \textsc{Hausdorff}-Abstand $d(x_0,C)$ zwischen $x_0$ und $C$ entspricht.
%  Wir wissen bereits, dass $D$ stetig ist. 
 Weiter betrachten wir für $x_1\in \R^d$
 \begin{align*}
  &\Vert P_C(x_1)-P_C(x_0)\Vert&=&\Vert (P_C(x_1)-x_1)-(P_C(x_0)-x_0)+(x_1-x_0)\Vert\\
  &&\leq& \Vert (P_C(x_1)-x_1)-(P_C(x_0)-x_0)\Vert+\Vert x_1-x_0\Vert
 \end{align*}

\end{compactenum}


