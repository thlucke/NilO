\assignment{1.2}

Es sei $f\colon\Rn\to\Rn$ eine Funktion.

% \begin{itemize}
% \item[(i)]
% Sei $\epi(f)$ konvex. Angenommen, $f$ ist keine konvexe Funktion. Dann
% gibt es Vektoren $x, y\in\Rn$ mit $x<y$, sowie ein $t \in [0,1]$ mit der Eigenschaft
% \[
% f(x+t (y-x)) > f(x) + t(f(y)-f(x)).
% \]
% Dann gilt aber $(x+t(y-x), f(x)+t(f(y)-f(x))) \notin \epi(f)$
% im Widerspruch zur Konvexit"at des Epigraphen.\\
%
% Sei nun $f$ eine konvexe Funktion. Betrachte f"ur $(x,\alpha), (y,\beta) \in\epi(f)$
% mit $x<y$ die Verbindungsstrecke $\gamma\colon[0,1] \to \epi(f)$,
% $t\mapsto (x,\alpha) + t((y,\beta)-(x,\alpha))$
% \end{itemize}

\begin{itemize}
 \item[(i)] Sei $f$ zus"atzlich konvex und seien $(x,\alpha),(y,\beta)\in\epi(f)$ sowie $\lambda\in[0,1]$ beliebig gew"ahlt.
 Dann folgt aus der Konvexit"at von $f$ und der Definition des Epigraphen
 \begin{displaymath}
  f(\lambda x+(1-\lambda)y)\leq \lambda f(x)+(1-\lambda)f(y)\leq \lambda\alpha+(1-\lambda)\beta.
 \end{displaymath}
 Daher ist der Epigraph von $f$ wegen
 \[
\lambda(x,\alpha)+(1-\lambda)(y,\beta)=
(\lambda x+(1-\lambda)y,\lambda\alpha+(1-\lambda)\beta)\in\epi(f)
 \]
 konvex.
 Ist hingegen $\epi(f)$ konvex und sind $x,y\in\Rn$ sowie $\lambda\in[0,1]$ vorgegeben, erhalten wir mit
 $(x,f(x)),(y,f(y))\in\epi(f)$ auch
\[
 (\lambda x+(1-\lambda)y,\lambda f(x)+(1-\lambda)f(y))\in \epi(f).
 \]
 Die Behauptung
 \begin{displaymath}
  f(\lambda x+(1-\lambda)y)\leq\lambda\alpha+(1-\lambda)\beta
 \end{displaymath}
 folgt sofort.

 \item[(ii)] Diese Aussage ist im Allgemeinen falsch, wie sich leicht an
 \begin{displaymath}
  f\colon\R\to\R, x\mapsto \begin{cases}
                       0, &x\in\R\setminus\{0\},\\
                       -1, &x=0,
                      \end{cases}
 \end{displaymath}
 einsehen l"asst. Offensichtlich ist $f$ nicht stetig. Dennoch ist
 \begin{displaymath}
  \epi(f)=\R\times[0,\infty)\cup\{0\}\times[-1,\infty)
 \end{displaymath}
 abgeschlossen.

 \item[(iii)]
 Wir zeigen zun"achst, dass $f$ genau dann unterhalbstetig ist, wenn $\epi(f)$ abgeschlossen ist.
 Zuerst wollen wir die Unterhalbstetigkeit von $f$ voraussetzen. Dann gilt
 \begin{displaymath}
  f(x)\leq\liminf_{t\to x}f(t).
 \end{displaymath}
 für alle $x\in\Rn$. Sei nun $(x_n,\alpha_n)_{n\in\N}\subset\epi(f)$ eine beliebige in $\Rn\times\R$ konvergente Folge.
 Wir setzen $x\coloneqq \lim_{n\to\infty} x_n$ und $\alpha\coloneqq\lim_{n\to\infty}\alpha_n$.
 Dies ist wohldefiniert, da Folgenkonvergenz in $\Rn\times\R$ "aquivalent zu komponentenweiser Konvergenz ist. Dann gilt
 wegen der Unterhalbstetigkeit von $f$
 \begin{displaymath}
  \alpha= \lim_{n\to\infty}\alpha_n \geq \liminf_{n\to\infty} f(x_n)\geq f(x)
 \end{displaymath}
 und somit $(x,\alpha)\in\epi(f)$. Es folgt die Abgeschlossenheit des Epigraphen von $f$.

 \newpage

 Ist nun $\epi(f)$ abgeschlossen und $(x_n)_{n\in\N}\subset \Rn$ eine beliebige in $\Rn$ konvergente Folge, so gilt
 wegen $(x_n,f(x_n))\in\epi(f)$ für alle $n\in\N$ und der Abgeschlossenheit von $\epi(f) $ auch
 \begin{align*}
 (x,\liminf_{n\to\infty}f(x_n)) &= (\lim_{n\to\infty}x_n, \liminf_{n\to\infty}f(x_n)) \\
&= (\liminf_{n\to\infty} x_n, \liminf_{n\to\infty}f(x_n)) \\
&\overset{(*)}{=} \liminf_{n\to\infty}(x_n, f(x_n)) \\
&= \lim_{n\to\infty}(x_n, f(x_n)) \in \epi(f).
 \end{align*}
 Bei $(*)$ haben wir wieder die Aquivalenz von komponentenweiser und Folgenkonvergenz
 genutzt.
 Es gilt demnach
 \begin{displaymath}
  f(x)\leq \liminf_{n\to\infty}f(x_n)
 \end{displaymath}
 womit die Aussage bewiesen ist. \hfill$\Box$\\

 Die eigentliche Behauptung folgt nun aus der Analysis 1 Übung vom 9. Dezember 2016. Dort wurde gezeigt, dass koerzitive und unterhalbstetige Funktionen ein
 Minimum besitzen. \hfill$\Box$
\end{itemize}
