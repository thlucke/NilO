\assignment{1.2}

Es sei $f\colon\Rn\to\Rn$ eine Funktion.

\begin{itemize}
\item[(i)]
Sei $\epi(f)$ konvex. Angenommen, $f$ ist keine konvexe Funktion. Dann
gibt es Vektoren $x, y\in\Rn$ mit $x<y$, sowie ein $t \in [0,1]$ mit der Eigenschaft
\[
f(x+t (y-x)) > f(x) + t(f(y)-f(x)).
\]
Dann gilt aber $(x+t(y-x), f(x)+t(f(y)-f(x))) \notin \epi(f)$
im Widerspruch zur Konvexit"at des Epigraphen.\\

Sei nun $f$ eine konvexe Funktion. Betrachte f"ur $(x,\alpha), (y,\beta) \in\epi(f)$
mit $x<y$ die Verbindungsstrecke $\gamma\colon[0,1] \to \epi(f)$,
$t\mapsto (x,\alpha) + t((y,\beta)-(x,\alpha))$
\end{itemize}
