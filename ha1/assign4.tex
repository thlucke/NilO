\assignment{1.4}
Wir "ubernehmen die Bezeichnungen aus der Aufgabenstellung und untersuchen die Matrix

\[
H:=
\begin{pmatrix}
H_1 & H_2 \\ H_2 & H_4 \end{pmatrix}
= 2 \begin{pmatrix}
\sum_{i=1}^m \xi_i^2 & \sum_{i=1}^m \xi_i \\
\sum_{i=1}^m \xi_i & m
\end{pmatrix}.
\]
\begin{compactenum}[(i)]
\item Zum Beweis der positiven Definitheit
bem"uhen wir das Hauptminorenkriterium: Da
nach Voraussetzung $\xi_i \neq \xi_j$ f"ur alle $i,j \in {1,\ldots,m}$ mit $i\neq j$ gilt, folgt unmittelbar
\[
\det(H_1) = 2\sum_{i=1}^m \xi_i^2 > 0.
\]
Wir setzen nun $\xi := (\xi_1,\ldots,\xi_m)\in\R^m$. Da bekannterma"sen die Absch"atzung
\[
\sqrt m\cdot \lVert \xi \rVert_2 \ge \lVert \xi \rVert_1
\]
gilt, folgt auch
\[
H_1 H_4 = 4 m \sum_{i=1}^m \xi_i^2
\ge 4\left(\sum_{i=1}^m \xi_i\right)^2
= H_2^2.
\]
Da nun alle Eintr"age von $\xi$ verschieden
sind, gilt sogar die strikte Ungleichung und
damit
\[
\det(H) = H_1 H_4 - H_2^2 > 0.
\]
Es folgt die positive Definitheit von $H$. \\ \\
Ferner ist $f$ zweimal stetig differenzierar und damit genau dann konvex, wenn $f''=H$ positiv definit ist. 
Hiermit folgt die Behauptung.
%Es gilt nach Vorlesung
% \begin{displaymath}
%  x^THx=\sum_{i=1}^m(x_1\xi_i+x_2)^2.
% \end{displaymath}
% Sei nun $x\in\R^2$ beliebig. Wegen $(x_1\xi_i+x_2)^2\geq0$ ist $x^THx\geq0$ und weil $\xi\mapsto x_1\xi+x_2$ nur eine Nullstelle besitzt, die $\xi_i$ aber
% paarweise verschieden sind, gibt es ein $\xi_i$ mit $(x_1\xi_i+x_2)^2>0$. Somit ist $x^THx>0$ für jedes $x\in\R^2$ und $H$ also positiv definit.
 %
 %\\

 \item Mit 1.1. wissen wir, dass $H$ sogar stark positiv ist. Wir haben also mit Cauchy-Schwarz und der starken Positivität
 ein $\alpha > 0$ mit
 \begin{align*}
  f(x)&=\frac{1}{2}\langle Hx,x\rangle+\langle b, x\rangle+c\\
  &\geq\frac{1}{2}\langle Hx,x\rangle-\Vert b\Vert_2\Vert x\Vert_2+c\\
  &\geq\frac{1}{2}\alpha\Vert x\Vert_2^2-\Vert b\Vert_2\Vert x\Vert_2+c \eqqcolon g(\Vert x\Vert_2).
 \end{align*}
 Wegen $\alpha >0$ folgt unmittelbar
 \begin{displaymath}
  \infty=\lim_{\Vert x\Vert_2\to\infty}g(\Vert x\Vert_2)\leq \lim_{\Vert x\Vert_2\to\infty} f(x).
 \end{displaymath}
 Somit ist $f$ koerzitiv und zusammen mit der Stetigkeit folgt die Existenz eines Minimierers. Aufgrund der strengen Konvexität von $f$ ist dieser eindeutig.

\end{compactenum}
