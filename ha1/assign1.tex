\assignment{1.1}

\begin{itemize}
\item[(i)]
Es sei $A\colon\Rn\to\Rn$ eine lineare Abbildung. Wir werden zeigen, dass
$A$ genau dann positiv definit ist, wenn $A$ stark positiv ist.

\begin{proof}
Sei $A$ zun"achst stark positiv. Dann existiert ein $\alpha>0$, sodass die
Absch"atzung
\begin{equation}\label{eq:starkPositiv}
\langle{Ax,x}\rangle \ge \alpha\lVert x\rVert^2
\end{equation}
f"ur alle $x\in\Rn$ erf"ullt ist. Der positiven Definitheit von Normen wegen,
folgt f"ur alle $x\in\Rn \setminus\{0\}$ unmittelbar
\[
\langle{Ax,x}\rangle \ge \alpha\lVert x\rVert^2 > 0.
\]
Die Abbildung $A$ ist also positiv definit.\\

Sei $A$ nun positiv definit und $x\in\Rn$ beliebig gew"ahlt. Ist $x=0$,
so gilt \eqref{eq:starkPositiv}
f"ur alle $\alpha>0$. Sei daher $x$ von Null verschieden. Dann gilt
\eqref{eq:starkPositiv} genau dann, wenn
\[
\frac{\langle Ax,x \rangle}{\lVert x \rVert^2} \ge \alpha
\]
beziehungsweise
\[
\langle Ay,y \rangle \ge \alpha
\]
f"ur alle $y\in\Rn$ mit $\lVert y \rVert = 1$ erf"ullt ist. Da lineare
Abbildungen -- auf endlichen Vekorr"aumen operierend -- beschr"ankt und
stetig sind, ist auch die Abbildung
\[
f\colon\mathbb{S}^1\to\R \text{, } y\mapsto\langle Ay,y\rangle
\]
stetig. Da die Einheitssph"are $\mathbb{S}^1$ kompakt ist, folgt aus dem
Satz von Weierstra"s die Existenz eines Minimums $\alpha\in\R$ mit
$f(y)\ge \alpha$ f"ur alle $y\in\mathbb{S}^1$. Da $A$ positiv definit ist,
muss zudem $\alpha>0$ gelten. Es folgt daher insgesamt
\[
\langle{Ax,x}\rangle \ge \alpha\lVert x\rVert^2
\]
f"ur alle $x\in\Rn$ und damit die starke Positivit"at von $A$.
\end{proof}

\item[(ii)]
Wir zeigen nun, dass (i) im Allgemeinen falsch ist. Dazu betrachten wir die
Abbildung
\[
A\colon \ell_2\to \ell_2 \text{, } (x_n)_{n\in\N} \mapsto
\left(\frac{1}{n}x_n\right)_{n\in\N}.
\]
Da f"ur alle $n\in\N$ die Absch"atzung $(x_n / n)^2 \le (x_n)^2$ gilt,
folgt wegen
\[
\sum_{n=1}^\infty \left(\frac{1}{n} x_n\right)^2 \le
\sum_{n=1}^\infty (x_n)^2
\]
die Wohldefiniertheit und Beschr"anktheit der Abbildung $A$ aus dem Majorantenkriterium f"ur Reihen.\\

Desweiteren folgt aus
\[
\langle A\seq{x}, \seq{x} \rangle = \sum_{n=1}^\infty \frac{1}{n}x_n^2 \ge 0
\]
die positive Definitheit. \\

Seien nun $\seq{x}, \seq{y}\in\ell_2$ zwei Folgen und $\lambda\in\R$ beliebig gew"ahlt.
Dann ist $A$ wegen
\begin{align*}
A(\lambda\seq{x}+\seq{y}) &= A(\seq{\lambda x} + \seq{y})\\
&= A((\lambda x_n + y_n)_{n\in\N})\\
&= \left(\frac{1}{n} (\lambda x_n + y_n) \right)_{n\in\N}\\
&= \left( \lambda\frac{x_n}{n} + \frac{y_n}{n}\right)_{n\in\N}\\
&= \lambda\left(\frac{x_n}{n}\right)_{n\in\N} + \left(\frac{y_n}{n}\right)_{n\in\N}\\
&= \lambda A\seq{x} + A\seq{y}
\end{align*}
linear.\\

Sei nun $\alpha > 0$ beliebig gew"ahlt. Dann gibt es ein $k\in\N$ mit $\frac{1}{k}<\alpha$.
F"ur den $k$-ten Einheitsbasisvektor $e_k=(\delta_{k,n})_{n\in\N}$ des $\ell_2$ gilt dann
\begin{displaymath}
 \langle Ae_k,e_k\rangle=\sum_{n=1}^\infty \frac{1}{n}\delta_{k,n}^2=\frac{1}{k}<\alpha.
\end{displaymath}
Demnach kann kein $\alpha > 0$ existieren, welches der Forderung
\[
\langle A(x_n)_{n\in\N},(x_n)_{n\in\N}\rangle \ge \alpha\lVert (x_n)_{n\in\N} \rVert^2
\]
f"ur alle $(x_n)_{n\in\N}\in\ell_2$ gen"ugt.

\end{itemize}
