\assignment{1.3}

Wir zeigen, dass ein normierter Raum ($X, \lVert\cdot\rVert)$ genau dann endlich
dimensional ist, wenn die Niveaumenge
$N(f,\alpha)$ f"ur jedes $\alpha\in\R$ und
jedes stetige koerzitive Funktional
$f\colon X\to\R$ kompakt ist. Dazu
nutzen wir im Folgenden die zum Hinweis äquivalente Aussage: Ein normierter Raum ist genau dann endlichdimensional, wenn beschränkte und abgeschlossene Mengen kompakt sind,
dies ist insbesondere genau dann erfüllt, wenn die abgeschlossene Einheitskugel kompakt ist.

\begin{proof}
Wir wollen zun"achst annehmen, dass $X$ endlichdimensional ist. Sei $f\colon X\to\R$
ein beliebig gew"ahltes stetiges und koerzitives Funktional und sei $\alpha\in\R$
vorgegeben.
 Dann gibt es ein
 $r\in\R$, sodass für alle $x\in X$ mit $\Vert x\Vert>r$ bereits $f(x)>\alpha$ gilt. Somit ist $N(\alpha):=N(f,\alpha) \subset \overline{B(0,r)}$ und es folgt die
 Beschr"anktheit der Niveaumenge $N(\alpha)$.\\

 Sei nun $(x_n)_{n\in\N}\subset N(\alpha)\subset X$ eine gegen $x\in X$ konvergente Folge. Aus der Stetigkeit von $f$ folgt
 \begin{displaymath}
  f(x)=\lim_{n\to\infty} f(x_n)  \le \lim_{n\to\infty} \alpha =\alpha.
 \end{displaymath}
und somit wegen $x\in N(\alpha)$ die Abgeschlossenheit der Niveaumenge $N(\alpha)$. Da $X$ endlichdimensional ist,
folgt wegen des Satzes von Heine-Borel die
Kompaktheit der Niveaumenge.\\

Sei nun für jedes $\alpha\in \R$ und jedes stetige, koerzive Funktional $f\colon X\to\R$ die Niveaumenge $N(\alpha)$ kompakt. Dann gilt
dies insbesondere für jede Norm $\Vert\cdot\Vert: X\to [0,\infty)$.
Dabei ist $\Vert\cdot\Vert$ offensichtlich koerziv und stetig. Weiterhin ist $N(1)=\overline{B(0,1)}$ nach Voraussetzung kompakt.
Dass die abgeschlossene Einheitskugel kompakt ist, ist äquivalent zur Endlichdimensionalit"at von $X$.
\end{proof}
