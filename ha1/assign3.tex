\assignment{1.3}

Wir nutzen im Folgenden die zum Hinweis äquivalente Aussage: Ein normierter Raum $X$ ist genau dann endlichdimensional, wenn beschränkte und abgeschlossene Mengen kompakt sind,
dies ist insbesondere genau dann erfüllt, wenn die abgeschlossene Einheitskugel kompakt ist.
\\\\
Sei $(X,\Vert\cdot\Vert)$ ein normierter Raum.\\
\begin{itemize}
 \item [$\Rightarrow$] Sei $X$ zus"atzlich endlichdimensional und $a\in\R$ beliebig.
 Sei ferner $f$ ein beliebiges stetiges und koerzives Funktional.
 Weil $f$ koerziv ist, gibt es ein 
 $r\in\R$, sodass für alle $x\in X$ mit $\Vert x\Vert>r$ bereits $f(x)>\alpha$ gilt. Somit ist $N(\alpha) \subset \overline{B(0,r)}$,
 d.h. $N(\alpha)$ ist beschränkt.\\
 Sei nun $(x_n)_{n\in\N}\subset N(\alpha)\subset X$ eine gegen $x\in X$ konvergente Folge. Weil $f$ stetig ist, gilt
 \begin{displaymath}
  f(x)=\lim_{n\to\infty} f(x_n) =\lim_{n\to\infty} \alpha =\alpha.
 \end{displaymath}
Somit ist auch $x\in N(\alpha)$ und die Niveaumenge abgeschlossen. Weil $X$ endlichdimensional ist, folgt mit Heine-Borel die Kompaktheit von $N(\alpha)$.\\
\item [$\Leftarrow$] Für jedes $\alpha\in \R$ und jedes stetige, koerzive Funktional $f$ sei die Niveaumenge $N(\alpha)$ kompakt. Dann gilt
dies insbesondere für $\Vert\cdot\Vert: X\to [0,\infty)$.
Dabei ist $\Vert\cdot\Vert$ offensichtlich koerziv und stetig. Weiterhin ist $N(1)=\overline{B(0,1)}$ nach Voraussetzung kompakt.
Dass die abgeschlossene Einheitskugel kompakt ist, ist äquivalent zur Endlichdimensionalit"at von $X$.
\end{itemize}
